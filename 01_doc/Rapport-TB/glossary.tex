\newglossaryentry{heig-vd}{
    name=HEIG-VD,
    description={Haute École d'Ingénierie et de Gestion du canton de Vaud}
}

\newglossaryentry{uml}{
    name=UML,
    description={Notation pour la modélisation d'applications en ingénieurie logiciel}
}

\newglossaryentry{diagrams}{
    name=diagrams.net,
    description={Site web permettant de créer des diagrammes}
}

\newglossaryentry{websig}{
    name=webSIG,
    description={systèmes d'interface web de visualisation de données géographiques}
}

\newglossaryentry{epfl}{
    name=EPFL,
    description={Ecole polytechnique fédéral de Lausanne}
}

\newglossaryentry{html}{
    name=HTML,
    description={Norme de fichier décrivant le contenu d'une page web}
}

\newglossaryentry{css}{
    name=CSS,
    description={Norme de fichier permettant de modifier l'aspect du contenu d'une page}
}

\newglossaryentry{js}{
    name=JavaScript,
    description={Language de programmation utilisé par les navigateurs web afin de modifier les contenus des applications web}
}

\newglossaryentry{rp}{
    name=reverse-proxy,
    description={Permet d'accéder à différents serveurs en utilisant le même url}
}

\newglossaryentry{container}{
    name=container,
    description={Entité contenant l'application ainsi que les autres programmes pour son fonctionnement}
}

\newglossaryentry{node}{
    name=Node,
    description={Programme permettant d'utiliser JavaScript en dehors des navigateurs web}
}

\newglossaryentry{npm}{
    name=NPM,
    description={Programme utilisé pour gérer les librairies Node}
}

\newglossaryentry{ts}{
    name=TypeScript,
    description={Language de programmation}
}

\newglossaryentry{vue3}{
    name=Vue 3,
    description={librairie JavaScript simplifiant les interractions avec le contenu}
}

\newglossaryentry{ol}{
    name=Openlayers,
    description={Librairie JavaScript permettant de créer des webSig}
}

\newglossaryentry{pinia}{
    name=Pinia,
    description={librairie associé à Vue 3 permettant de simplifier la communication de données entre les différents component}
}

\newglossaryentry{vite}{
    name=Vite,
    description={Un environnement de développement}
}

\newglossaryentry{vitest}{
    name=Vitest,
    description={librairie permettant de créer des tests unitaires}
}

\newglossaryentry{vuetest}{
    name=Vue test Utils v2,
    description={librairie permettant de tester les components Vue}
}

\newglossaryentry{express}{
    name= Express,
    description={Librairie siplifiant la mise en place d'un serveur sur Node}
}

\newglossaryentry{traefik}{
    name=Traefik,
    description={Programme servant de reverse-proxy}
}

\newglossaryentry{nginx}{
    name=Nginx,
    description={Serveur Http}
}

\newglossaryentry{postgres}{
    name=Postgres,
    description={Base de données relationnelle}
}

\newglossaryentry{postgis}{
    name=PostGis,
    description={Extension de Postgres rajoutant la possibilité de stocker des données géographiques}
}

\newglossaryentry{git}{
    name=Git,
    description={Programme de versionning}
}

\newglossaryentry{gitlab}{
name=GitLab
description={Hébergeur de code source}
}

\newglossaryentry{cicd}{
    name=GitLab CI/CD,
    description={Outil permettant la mise en place d'un pipeline CI/CD}
}

\newglossaryentry{registry}{
    name=GitLab container registry,
    description={Hébergeur d'image de container}
}

\newglossaryentry{docker}{
    name=Docker,
    description={Programme gérant les container}
}

\newglossaryentry{qgis}{
    name=QGis,
    description={logiciel de système d'information géographique}
}

\newglossaryentry{geojson}{
    name=GeoJson,
    description={Norme pour les fichiers Json permettant de stocker des données géographiques}
}